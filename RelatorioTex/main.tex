\documentclass[
 paper=A4,               % paper size --> A4 is default in Germany
    twoside=true,           % onesite or twoside printing
    openright,              % doublepage cleaning ends up right side
    parskip=full,           % spacing value / method for paragraphs
    chapterprefix=true,     % prefix for chapter marks
    12pt,                   % font size
    headings=normal,        % size of headings
    bibliography=totoc,     % include bib in toc
    listof=totoc,           % include listof entries in toc
    titlepage=on,           % own page for each title page
    captions=tableabove,    % display table captions above the float env
    draft=false,            % value for draft version
]{scrreprt}

%\usepackage[utf8x]{inputenc}
\usepackage[utf8]{inputenc}
\usepackage[T1]{fontenc}
\usepackage{newstyle}
\usepackage{tabularx}
\usepackage{amsmath,amsfonts,amssymb,amsthm,url}
\def\UrlBreaks{\do\/\do-}
\usepackage[portuguese,english]{babel}
\usepackage{times}
\usepackage{setspace}
\usepackage[sort&compress,numbers]{natbib}
\usepackage[vlined,ruled,commentsnumbered,linesnumbered]{algorithm2e} 
\usepackage{algorithmic}  
\usepackage{graphicx,multirow}
\usepackage{array}
\usepackage{multicol}
\usepackage{mathptmx} % use Times in math mode
%\usepackage{moreverb}
\usepackage{lscape,rotating}
\usepackage{hyperref}
\usepackage{fancyhdr}
\usepackage{lastpage}
\usepackage{nameref}
\usepackage{listings}
\usepackage{subfig}
\usepackage{tikz}
\usepackage{xcolor}
\usepackage{makeidx}
\usepackage{epsfig}
\usepackage{xspace}
%\usepackage{subfigure}
%\usepackage[show]{chato-notes}
%\usepackage{listings}
\usepackage{color}
\usepackage{marginnote}
\usepackage{flushend}
\usepackage{breakurl}
\usepackage{lipsum}  
\usepackage{longtable}
\usepackage[acronym]{glossaries}

%\makeindex


\theoremstyle{definition}
\newtheorem{defn}{Definition}[]

\newcommand*\circled[1]{\tikz[baseline=(char.base)]{
            \node[shape=circle,fill,inner sep=1pt] (char) {\textcolor{white}{#1}};}}

\fancyhf{} %
\lhead{\nouppercase {\leftmark}} %
\rhead{\nouppercase {\bf \thepage}}
\renewcommand{\headrulewidth}{0.1pt}

% Comando para inserir pagina em branco (inserida na numeracao, mas sem
% numero impresso) para quando e' preciso obrigar um capitulo a comecar
% do lado direito (pagina impar)
\newcommand{\LIMPA}{
\newpage
\mbox{}
\thispagestyle{empty}
}


\newcommand{\Cabecalho}{
\vspace{-0.5cm}\normalfont\normalfont
\vfill
\textsc{\normalsize\uppercase{Instituto Superior Manuel Teixeira Gomes}}\\
\vspace{1.5cm}
\includegraphics[scale=.9]{pic/logotipo_ISMAT.pdf}\\
}

%%%%%%%%%%%%%%%%%%%%%%%%%%%%%%%%%%%%%%%%%%%%%%%%%%%%%%%%%%%%%%%%%%%%%%%%%
% 						INFORMACOES RELATIVAS AO PROJECTO/ALUNO										%
%%%%%%%%%%%%%%%%%%%%%%%%%%%%%%%%%%%%%%%%%%%%%%%%%%%%%%%%%%%%%%%%%%%%%%%%%


%%%%%%%%%%%%%%%%%%%%%%%%%%%%%%%%%%%%%%%%%%%%%%%%%%%%%%%%%%%%%%%%%%%%%%%%%
% 				ALTERAR AQUI AS INFORMACOES RELATIVAS AO PROJECTO							%
%%%%%%%%%%%%%%%%%%%%%%%%%%%%%%%%%%%%%%%%%%%%%%%%%%%%%%%%%%%%%%%%%%%%%%%%%
\newcommand{\TITULO}{BlockChain}
\newcommand{\Autor}{Paulo Craveiro}
\newcommand{\AutorNumAluno}{a21700285}
\newcommand{\Autorr}{Pedro Roldan}
\newcommand{\AutorNumAlunoo}{a21501217}
\newcommand{\Autorrr}{Márcio Silva}
\newcommand{\AutorNumAlunooo}{a21802342}
\newcommand{\MIC}{Relatório de Computação Distribuída}
\newcommand{\PFC}{Relatório de Trabalho Final de Curso}

%Orientador e CoOrientador *sem* titulos (e.g. Prof. Doutor)
\newcommand{\Orientador}{}
\newcommand{\CoOrientador}{} %se nao se aplicar, nao importa o que aqui esteja

%Se aplicavel, o supervisor pode ter um titulo (Dr., Eng.) colocado aqui
\newcommand{\SupervisorInstituicao}{Doutor Miguel Garcia Henriques}  %se nao se aplicar, nao importa o que aqui esteja

\newcommand{\AnoLectivo}{2019/2020}
\newcommand{\Ano}{\Large{2020}}

% Comentar/descomentar conforme conveniente
\newcommand{\CADEIRA}{\MIC}

% Comentar/descomentar conforme conveniente
\newcommand{\IdiomaTese}{\selectlanguage{portuguese}}
%\newcommand{\IdiomaTese}{\selectlanguage{english}}
%%%%%%%%%%%%%%%%%%%%%%%%%%%%%%%%%%%%%%%%%%%%%%%%%%%%%%%%%%%%%%%%%%%%%%%%%%




%%%%%%%%%%%%%%%%%%%%%%%%%%%%%%%%%%%%%%%%%%%%%%%%%%%%%%%%%%%%%%%%%%%%%%%%%%



\title{\TITULO}
\author{\Autor}
%\date{\today}

\makeglossaries

\begin{document}
\selectlanguage{portuguese}
\pagestyle{empty}

%%% Primeira capa
\begin{center}

\Cabecalho

\vspace{2.0cm}
\vfill
\IdiomaTese
\Large{\textbf{\TITULO}}\\
\vspace{1cm}
\vfill

\large{\textbf{\CADEIRA}}\\
\vspace{1.0cm}
\vfill
\Large{\textbf{\Autor}}\\
\vspace{1,8 cm}
\vfill
\large{Trabalho orientado por:}\\
\large{Prof. \Orientador} \\
% DESCOMENTAR a linha relevante (se alguma), removendo o % no inicio
e pelo Prof. \CoOrientador \\
\vspace{1 cm}
\vfill

\vspace{0.5cm}
\vfill
\vspace{-1cm}
\Ano
\end{center}


%%% Segunda capa
\begin{center}
\newpage
\thispagestyle{empty}
\mbox{}
\newpage
\cleardoublepage

\Cabecalho

\vspace{2.0cm}
\vfill
\IdiomaTese
\Large{\textbf{\TITULO}}\\
\vspace{1cm}
\vfill

\large{\textbf{\CADEIRA}}\\
\vspace{1.0cm}
\vfill
\Large{\textbf{\Autor}}\\
\vspace{1,8 cm}
\vfill
\large{Trabalho orientado por:}\\
\large{Prof. \Orientador} \\
% DESCOMENTAR a linha relevante (se alguma), removendo o % no inicio
e pelo Prof. \CoOrientador \\
\vspace{1 cm}
\vfill

\vspace{0.5cm}
\vfill
\vspace{-1cm}
\Ano
\end{center}
\newpage
\thispagestyle{empty}
\mbox{}
\newpage
\cleardoublepage


% Fim da capa
% ----------------------------------------------------------------------

\pagenumbering{roman}			% roman page numbing (invisible for empty page style)
\setcounter{page}{1}			% set page counter
\pagestyle{empty}

\selectlanguage{portuguese}

\pagestyle{plain}				% display just page numbers
% ----------------------------------------------------------------------
\chapter*{\TITULO{}  Declaração de autoria}
\label{chapter:copyright}

Declaro ser o (a) autor(a) do trabalho apresentado neste relatório, sendo original e inédito. Autores e trabalhos consultados estão devidamente citados no texto e constam da listagem de referências incluída.
\vspace*{15mm}

\begin{center}
\underline{\hspace{4.5cm}}\\
(Assinatura)
\end{center}
\vspace*{20mm}


O Instituto Superior Manuel Teixeira Gomes tem o direito, perpétuo e sem limites geográficos, de arquivar e publicitar este trabalho através de quaisquer meios, de o divulgar através de repositórios científicos e de admitir a sua cópia e distribuição com objetivos não comerciais, desde que seja dado crédito ao autor.
\vspace*{15mm}
\begin{center}
\underline{\hspace{4.5cm}}\\
(Assinatura)
\end{center}

% ----------------------------------------------------------------------

%\LIMPA

\pagestyle{plain}				% display just page numbers
%\pagestyle{plain}

\vspace*{2cm}

\chapter*{Agradecimentos}

\lipsum[1-3]




% ----------------------------------------------------------------------

\LIMPA

\pagestyle{plain}				% display just page numbers
% ----------------------------------------------------------------------

\chapter*{Resumo}
\label{chapter:abstract_pt}
\vspace*{-10mm}

isto é um resumo de teste peço desculpa

\vfill

\begin{flushleft}
\textbf{Palavras-chave:}
Keyword1, Keyword1, Keyword3, Keyword4, Keyword5.
\end{flushleft}

%\LIMPA


\LIMPA

\pagestyle{plain}				% display just page numbers
% ----------------------------------------------------------------------
\selectlanguage{english}

\chapter*{Abstract}
\label{chapter:abstract_en}
\vspace*{-10mm}
\lipsum[1-3]

\vfill

\begin{flushleft}
\textbf{Keywords:}
Keyword1, Keyword1, Keyword3, Keyword4, Keyword5.
\end{flushleft}
\selectlanguage{portuguese}
%\LIMPA

% ----------------------------------------------------------------------

\LIMPA

%Lista de capitulos
\setcounter{secnumdepth}{3}
\setcounter{tocdepth}{2}		% define depth of toc
\tableofcontents				% display table of contents
%\addcontentsline {toc} {chapter} {Content}
\newpage
\thispagestyle{empty}
\mbox{}
\newpage
%\cleardoublepage

%%Lista de figuras
\listoffigures
%\addcontentsline {toc} {chapter} {List of Figures}
\newpage
\thispagestyle{empty}
\mbox{}
\newpage
%\cleardoublepage

%Lista de tabelas
\listoftables
%\addcontentsline {toc} {chapter} {List of Tables}
\newpage
\thispagestyle{empty} 
\mbox{}
\newpage
\cleardoublepage


%Lista de acronimos
%\LIMPA
\renewcommand{\glossaryname}{Acrónimos}
% abbreviations:
\newacronym{cd}{CD}{Computação Distribuída}

\clearpage
\printglossary[type=\acronymtype]
 

% --------------------------
% Body matter
% --------------------------

% ----------------------------------------------------------------------
% Inicio conteudo
\pagestyle{fancy}
\cleardoublepage

\selectlanguage{portuguese}
\pagenumbering{arabic}			% arabic page numbering
\setcounter{page}{1}			% set page counter
\pagestyle{maincontentstyle} 		% fancy header and footer

%%% ADICIONAR CAPITULOS
%%% ADICIONAR CAPITULOS
\chapter{Introdução}
\label{chap:introduction}

Teste peço.


This is a citation example~\cite{Garcia:2018}

\gls{nasa} is an acronym example. \gls{nasa} is cool.

\begin{figure}[h]
\begin{center}
\includegraphics[width=\columnwidth]{images/figure_example.pdf}
%\vspace{-5mm}
\caption{This is figure example.}
\label{fig:overview}
\end{center}
\end{figure}



This is an equation Example
\begin{equation} 
\begin{split}
\textit{oldness}(v)=\\\textit{max}\left((1-0.25\times\frac{(\textit{now}-v.\textit{published\_date})}{\textit{oldness\_threshold}}), 0.75\right)
\label{eq:oldness}
\end{split}
\end{equation}

\begin{algorithm}[H]
\SetAlgoLined
\KwResult{Write here the result }
 initialization\;
 \While{While condition}{
  instructions\;
  \eIf{condition}{
   instructions1\;
   instructions2\;
   }{
   instructions3\;
  }
 }
 \caption{Algorithm example}
\end{algorithm}



\begin{table}[!t]
\begin{center}
{\footnotesize
\begin{tabular}{| l | l | c | c |}\hline
\textbf{ID} & \textbf{Name}  & \textbf{\#Cores} & \textbf{Memory}  \\\hline\hline
UB14 & Ubuntu 14.04 & 4 & 15GB  \\ \hline
UB16 & Ubuntu 16.04 & 4 & 15GB \\ \hline
UB17 & Ubuntu 17.04 & 4 & 15GB\\ \hline
OS42 & OpenSuse 42.1 & 4 & 15GB \\ \hline
FE24 & Fedora 24 & 4 & 15GB  \\ \hline
FE25 & Fedora 25 & 4 & 15GB  \\ \hline
FE26 & Fedora 26 & 4 & 15GB \\ \hline
\end{tabular}
}
\caption{TABLE EXAMPLE}
\label{tab:oses}
\end{center}
\end{table}



\chapter{Desenvolvimento do Sistema}
\label{chap:systemdevelopment}

Teste 2.
\chapter{Implementação e Testes}
\label{chap:testingimplementation}

Teste 3.

\chapter{Considerações Finais}
\label{chap:finalconsiderations}

Teste 4.

%%% \include{chapters/introduction/CAPITULO}
%%% \include{chapters/introduction/CAPITULO}



% Fim do conteudo
% ----------------------------------------------------------------------

% Glossario

%\LIMPA
\bibliographystyle{IEEEtran}
\bibliography{chapters/references/references}
%\addcontentsline {toc} {chapter} {Bibliography}

\end{document}

